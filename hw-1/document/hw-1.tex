\documentclass[12pt]{article}

% ========== PACKAGES ========== 
% Math and equations
\usepackage{amsmath}
\usepackage{amssymb}

% Graphics and images
\usepackage{graphicx}
\usepackage{float}

% Tables
\usepackage{booktabs}
\usepackage{tabularx}
\usepackage{array}

% Layout and formatting
\usepackage{geometry}
\usepackage{setspace}
\usepackage{fancyhdr}
\usepackage{tcolorbox}
\usepackage{xcolor}
\usepackage{pdflscape}  % For landscape orientation
\usepackage{xcolor}     % For colors
\usepackage{colortbl}   % For colored table rows

% Other utilities
\usepackage{hyperref}

% ========== DOCUMENT SETUP ========== 
% Margins (1 inch on all sides)
\geometry{margin=1in}

% Single spacing
\singlespacing

% Header and footer
\pagestyle{fancy}
\fancyhf{}
\rhead{\thepage}

% ========== COVER PAGE ========== 
\begin{document}

\begin{titlepage}
    \centering
    \vspace*{2cm}
    
    % University/Institution (if applicable)
    % {\large University Name \\[0.5cm]}
    
    {\LARGE\bfseries Advanced Thermodynamics\\[0.5cm]}
    {\Large Homework Assignment \#1\\[2cm]}
    
    % Problem summary box
    % \begin{tcolorbox}[colback=blue!5!white,colframe=blue!75!black,width=0.8\textwidth]
    %     \centering
    %     \textbf{Assignment Overview}\\[0.5cm]
    %     Problems 1-14: Fundamental thermodynamic concepts, energy balances, entropy analysis, steam turbines, Carnot cycles, and concentrated solar power optimization
    % \end{tcolorbox}
    
    % \vspace{2cm}
    
    % Student information
    {\large \textbf{Student:} Dominic LaVigne \\[1cm]}
    
    % Course information (if applicable)
    % {\large \textbf{Course:} ME XXX - Advanced Thermodynamics \\[0.5cm]}
    % {\large \textbf{Instructor:} Prof. [Name] \\[1cm]}
    
    {\large \textbf{Submitted:} February 2, 2026 \\[2cm]}
    
    % Computational tools box
    % \begin{tcolorbox}[colback=green!5!white,colframe=green!75!black,width=0.8\textwidth]
    %     \centering
    %     \textbf{Computational Tools Developed}\\[0.5cm]
    %     • CoolProp CLI for multi-fluid property lookup\\
    %     • Property solver for iterative thermodynamic calculations\\
    %     • Carnot cycle optimization analysis\\
    %     • Steam turbine performance analysis
    % \end{tcolorbox}
    
    % \vspace{\fill}
\end{titlepage}

\newpage

% Table of contents
\tableofcontents
\newpage







% ========== PROBLEM-BY-PROBLEM HOMEWORK TEMPLATE ========== 

% Instructions: For each homework problem, copy the block below and increment the problem number.
% Each problem will start on a new page.
\newpage
\addcontentsline{toc}{section}{General Theory}
\vspace{0.5cm}
% Write your solution to Problem 1 here.
\subsection*{1.}
It is assumed the system is a simple compressible system, therefore there are two degrees of freedom and the rest of the intensive properties can be calculated using an equation of state, or found in a table of common fluids.

\section*{2.}
\begin{itemize}
    \item Closed system: (a)
    \item Open system: (c)
    \item Isolated system: (b)
\end{itemize}
\section*{3.}
\textbf{Answer:}
\begin{enumerate}
    \item[a.] internal energy
    \item[b.] kinetic energy
    \item[c.] potential energy
\end{enumerate}

\textbf{Notes:}
\begin{itemize}
    \item[f.] Electrical energy could potentially be considered if the mass crossing a system boundary is electrically charged, but it would be a rare circumstance.
    \item[e.] Flow work is a summation of other components, not necessarily carried by the mass itself.
    \item[c.] $E = mc^2$, so an argument could be made but it is not relevant here.
\end{itemize}
\section*{4.}
\textbf{False.} Only solids resist shear stress by static deflection. Fluids continuously deform under shear stress.
\section*{5.}
(c), assuming $V_n$ is the velocity normal to $A_c$, then the final units are kg/s.

\newpage

\section*{Problem 6: Hydro Storage System Energy Analysis}
\addcontentsline{toc}{section}{Problem 6: Hydro Storage System Energy Analysis}

\textbf{Given:}
\begin{itemize}
    \item Volumetric flow rate: $\dot{V} = 500$ m$^3$/s
    \item Elevation drop: $\Delta z = 85$ m
    \item Water density: $\rho = 1000$ kg/m$^3$ (assumed)
\end{itemize}

\begin{enumerate}
    \item[(a)] \textbf{Maximum possible power with upper reservoir as a lake ($V_1 = 0$):}
    \begin{itemize}
        \item \textbf{Apply the Energy Equation:}
        \begin{align*}
            \frac{V_1^2}{2} + gz_1 &= \frac{V_2^2}{2} + gz_2 + w_{\text{turbine}} \\
            \text{where:} \\
            V_1 &= \text{velocity at upper reservoir (lake)} = 0 \\
            V_2 &= \text{velocity at lower lake (assume 0)} \\
            z_1 &= \text{elevation of upper reservoir} \\
            z_2 &= \text{elevation of lower lake} \\
            g &= 9.81\ \mathrm{m/s^2}
        \end{align*}
        \item \textbf{Solve for work per unit mass:}
        \begin{align*}
            w_{\text{turbine}} &= g(z_1 - z_2)
        \end{align*}
        \item \textbf{Calculate total power:}
        \begin{align*}
            \dot{W}_{\text{turbine}} &= \dot{m} \cdot w_{\text{turbine}} \\
            \dot{m} &= \rho Q = 1000\ \mathrm{kg/m^3} \times 500\ \mathrm{m^3/s} = 500{,}000\ \mathrm{kg/s} \\
            w_{\text{turbine}} &= 9.81\ \mathrm{m/s^2} \times 85\ \mathrm{m} = 833.85\ \mathrm{J/kg} \\
            \dot{W}_{\text{turbine}} &= 500{,}000\ \mathrm{kg/s} \times 833.85\ \mathrm{J/kg} = 416,925,000\ \mathrm{W} = \boxed{416.9\ \mathrm{MW}}
        \end{align*}
    \end{itemize}

    \item[(b)] \textbf{Maximum possible power with upper reservoir as a river ($V_1 = 3.75\ \mathrm{m/s}$):}
    \begin{itemize}
        \item \textbf{Include kinetic energy at inlet:}
        \begin{align*}
            w_{\text{turbine}} &= \frac{V_1^2}{2} + g(z_1 - z_2) \\
            &= \frac{(3.75\ \mathrm{m/s})^2}{2} + 9.81\ \mathrm{m/s^2} \times 85\ \mathrm{m} \\
            &= 7.03\ \mathrm{J/kg} + 833.85\ \mathrm{J/kg} = 840.88\ \mathrm{J/kg} \\
            \dot{W}_{\text{turbine}} &= 500{,}000\ \mathrm{kg/s} \times 840.88\ \mathrm{J/kg} = 420,440,000\ \mathrm{W} = \boxed{420.4\ \mathrm{MW}}
        \end{align*}
        \item \textbf{Percent increase in power:}
        \begin{align*}
            \%\ \text{increase} &= \frac{420.4 - 416.9}{416.9} \times 100\% = \boxed{0.84\%} 
        \end{align*}
        \item \textbf{Physical explanation:}\\
        The increase in possible power output is due to the additional kinetic energy of the water entering the turbine when the upper reservoir is a river. This kinetic energy is converted to useful work by the turbine, resulting in a slightly higher power output compared to the case where the upper reservoir is a still lake.
    \end{itemize}
\end{enumerate}



\section*{Problem 7: Piston-Cylinder Energy Balance}
\addcontentsline{toc}{section}{Problem 7: Piston-Cylinder Energy Balance}

\textbf{Given:}
\begin{itemize}
    \item Mass of H$_2$ gas: $m = 3.8$ kg
    \item Gas constant: $R = 4.124$ kJ/kg$\cdot$K  
    \item Initial state: $P_1 = 250$ kPa, $T_1 = 900$ K
    \item Final temperature: $T_2 = 400$ K
    \item Process: Constant pressure (heat removal)
\end{itemize}

\textbf{Part A: Find the change in volume ($\mathrm{m}^3$) of the system.}

Using the ideal gas law to find the initial volume:
\begin{align*}
    PV &= mRT \\
    V_1 &= \frac{mRT_1}{P} \\
    &= \frac{(3.8\ \mathrm{kg})(4.124\ \mathrm{kJ/kg\cdot K})(900\ \mathrm{K})}{250\ \mathrm{kPa}} \\
    &= \boxed{56.416\ \mathrm{m}^3}
\end{align*}

For a constant pressure process, $\frac{T_1}{V_1} = \frac{T_2}{V_2}$:
\begin{align*}
    \frac{900}{56.416} &= \frac{400}{V_2} \\
    V_2 &= \frac{400 \times 56.416}{900} = \boxed{25.073\ \mathrm{m}^3}
\end{align*}

The change in volume is:
\begin{align*}
    \Delta V &= V_2 - V_1 = 25.073 - 56.416 = \boxed{-31.343\ \mathrm{m}^3}
\end{align*}

The negative sign indicates the volume decreased as the gas cooled.

\textbf{Part B: Find the magnitude (kJ) and direction (in or out of system) of the boundary work during this process. Defend why your answer makes sense.}

For a constant pressure process, the boundary work is:
\begin{align*}
    W &= P\Delta V = P(V_2 - V_1) \\
    &= 250\ \mathrm{kPa} \times (25.073 - 56.416)\ \mathrm{m}^3 \\
    &= 250\ \mathrm{kPa} \times (-31.343\ \mathrm{m}^3) \\
    &= \boxed{-7,836\ \mathrm{kJ}}
\end{align*}

The negative sign indicates work is done \textbf{by} the system (the gas does work on the surroundings as it contracts). If you define work done \textbf{on} the system, $W = +7,836\ \mathrm{kJ}$ into the system. This makes sense because as heat is removed at constant pressure, the gas contracts and the surroundings do work on the gas.

\textbf{Part C: Calculate the change in specific enthalpy ($\mathrm{kJ/kg}$) and internal energy ($\mathrm{kJ/kg}$) of the gas during this process using the given property tables.}

\begin{itemize}
    \item \textbf{Change in specific enthalpy ($\Delta h$):}
    \begin{align*}
        \Delta h &= \int_{T_1}^{T_2} C_p(T)\,dT \\
        &\approx \bar{C_p} \times (T_2 - T_1)
    \end{align*}
    Where $\bar{C_p}$ is the average of $C_p$ at 900 K and 400 K (read from the graph):
    \begin{align*}
        \bar{C_p} &\approx \frac{14.35 + 14.93}{2} = 14.64\ \mathrm{kJ/kg\cdot K} \\
        \Delta h &= 14.64\ \mathrm{kJ/kg\cdot K} \times (400 - 900)\ \mathrm{K} \\
        &= 14.64 \times (-500) = \boxed{-7,320\ \mathrm{kJ/kg}}
    \end{align*}
    \item \textbf{Change in specific internal energy ($\Delta u$):}
    \begin{align*}
        \Delta u &= \int_{T_1}^{T_2} C_v(T)\,dT \\
        &\approx \bar{C_v} \times (T_2 - T_1)
    \end{align*}
    Where $\bar{C_v}$ is the average of $C_v$ at 900 K and 400 K (read from the graph, e.g. $10.2$ and $10.9$):
    \begin{align*}
        \bar{C_v} &\approx \frac{10.2 + 10.9}{2} = 10.55\ \mathrm{kJ/kg\cdot K} \\
        \Delta u &= 10.55\ \mathrm{kJ/kg\cdot K} \times (400 - 900)\ \mathrm{K} \\
        &= 10.55 \times (-500) = \boxed{-5,275\ \mathrm{kJ/kg}}
    \end{align*}
    \item Both $\Delta h$ and $\Delta u$ are negative, indicating the gas lost enthalpy and internal energy as it cooled from 900 K to 400 K.
\end{itemize}
\newpage

\section*{Problem 8: Centrifugal Pump Energy Balance}
\addcontentsline{toc}{section}{Centrifugal Pump Energy Balance}
\vspace{0.5cm}

\textbf{Given:}
\begin{itemize}
    \item Shaft work: $\dot{W}_s = 10\ \mathrm{kW}$
    \item Mass flow rate: $\dot{m} = 0.05\ \mathrm{kg/s}$
    \item Water properties: $\rho = 1000\ \mathrm{kg/m^3}$, $C = 4.2\ \mathrm{kJ/kg\cdot K}$
    \item Inlet temperature: $T_1 = 20^\circ\mathrm{C}$
    \item Outlet temperature: $T_2 = 45^\circ\mathrm{C}$
    \item Cross-sectional areas: $A_{inlet} = A_{outlet}$
\end{itemize}

\textbf{Part A: Prove that the change in kinetic energy is negligible}

\begin{itemize}
    \item \textbf{Given:}
    \begin{align*}
        \dot{m} &= 0.05\ \mathrm{kg/s} \\
        A_2 &= A_3 \\
        \rho &= 1000\ \mathrm{kg/m^3}
    \end{align*}
    \item \textbf{Conservation of mass for steady flow:}
    \begin{align*}
        \dot{m} &= \rho A_2 v_2 = \rho A_3 v_3 \\
        \Rightarrow v_2 = v_3
    \end{align*}
    \item \textbf{Change in kinetic energy:}
    \begin{align*}
        \Delta KE &= \frac{1}{2} m (v_3^2 - v_2^2) = 0
    \end{align*}
    \item \textbf{Conclusion:} The change in kinetic energy is negligible (zero) for the water in this pump.
\end{itemize}

\textbf{Part B: Estimate the heat loss (kW) from the pump}

Applying the flow energy equation:
\begin{align*}
    \dot{Q}_{\text{loss}} - \dot{W}_s &= \dot{m}\left[\Delta h + \Delta KE + \Delta PE\right]
\end{align*}

With the assumptions that $\Delta KE = 0$ (from Part A), $\Delta PE \approx 0$ (not significant), and for incompressible liquid $\Delta h = C \Delta T$, the energy balance becomes:
\begin{align*}
    \dot{Q}_{\text{loss}} - \dot{W}_s &= \dot{m} \cdot C \cdot (T_2 - T_1) \\
    \dot{m} \cdot C \cdot \Delta T &= 0.05 \times 4.2 \times (45 - 20) \\
    &= 0.05 \times 4.2 \times 25 \\
    &= 5.25\ \mathrm{kW}
\end{align*}

Solving for the heat loss:
\begin{align*}
    \dot{Q}_{\text{loss}} &= \dot{W}_s - \dot{m} \cdot C \cdot \Delta T \\
    &= 10 - 5.25 = \boxed{4.75\ \mathrm{kW}}
\end{align*}

Therefore, the heat loss from the pump is approximately $4.75\ \mathrm{kW}$.

\newpage

\section*{Problem 9: Isentropic Steam Turbine with Extraction}
\addcontentsline{toc}{section}{Isentropic Steam Turbine}
\vspace{0.5cm}

\textbf{Given:}
\begin{itemize}
    \item Steam inlet: $\dot{m}_1 = 3\ \mathrm{kg/s}$ at $P_1 = 3\ \mathrm{MPa}$ (saturated vapor)
    \item Extraction: 12\% of flow at $P_2 = 500\ \mathrm{kPa}$  
    \item Exhaust: 88\% of flow at $P_3 = 50\ \mathrm{kPa}$, $T_3 = 100^\circ\mathrm{C}$
    \item Process: Isentropic (reversible adiabatic)
\end{itemize}
\textbf{Solution:}

The mass flow rates are $\dot{m}_2 = 0.12 \times \dot{m}_1 = 0.36\ \mathrm{kg/s}$ and $\dot{m}_3 = \dot{m}_1 - \dot{m}_2 = 2.64\ \mathrm{kg/s}$.

The steady flow energy equation for the turbine is:
\begin{align*}
    \dot{W}_{\text{out}} = \dot{m}_1 h_1 - \dot{m}_2 h_2 - \dot{m}_3 h_3
\end{align*}
where $h_1$ is the specific enthalpy at inlet (3 MPa, saturated vapor), $h_2$ is at extraction (500 kPa), and $h_3$ is at exhaust (50 kPa, 100$^\circ$C).

\textbf{Steam Properties from CoolProp:}
\begin{itemize}
    \item \textbf{State 1 (Inlet):} $P_1 = 3$ MPa, saturated vapor
    \begin{align*}
        T_1 &= 507.0\ \mathrm{K} = 234^\circ\mathrm{C} \\
        h_1 &= 2803.2\ \mathrm{kJ/kg} \\
        s_1 &= 6185.6\ \mathrm{J/kg \cdot K} = 6.186\ \mathrm{kJ/kg \cdot K}
    \end{align*}
    \item \textbf{State 2 (Extraction):} $P_2 = 500$ kPa, $s_2 = s_1$ (isentropic)
    \begin{align*}
        s_2 &= 6185.6\ \mathrm{J/kg \cdot K} \\
        h_2 &= 2478.2\ \mathrm{kJ/kg}
    \end{align*}
    \item \textbf{State 3 (Exhaust):} $P_3 = 50$ kPa, $T_3 = 100^\circ$C
    \begin{align*}
        h_3 &= 2682.4\ \mathrm{kJ/kg}
    \end{align*}
\end{itemize}

The power output is:
\begin{align*}
    \dot{W}_{\text{out}} &= \dot{m}_1 h_1 - \dot{m}_2 h_2 - \dot{m}_3 h_3 \\
    &= (3.0)(2803.2) - (0.36)(2478.2) - (2.64)(2682.4) \\
    &= 8409.6 - 892.2 - 7081.5 \\
    &= \boxed{435.9\ \mathrm{kW}}
\end{align*}


\newpage

\section*{Problem 10: Real Steam Turbine Entropy Analysis}
\addcontentsline{toc}{section}{Real Devices and Entropy Balances}
\vspace{0.5cm}

For this real steam turbine analysis, we have a mass flow rate of $\dot{m} = 2.5\ \mathrm{kg/s}$ with inlet conditions of $P_1 = 3\ \mathrm{MPa}$ and $T_1 = 400^\circ\mathrm{C}$ (superheated steam). The steam expands to an outlet pressure of $P_2 = 100\ \mathrm{kPa}$ with an isentropic efficiency of $\eta_s = 0.85$. The process is adiabatic expansion ($\dot{Q} = 0$) with negligible kinetic and potential energy changes.

\textbf{PART A: Isentropic Operation}

\begin{itemize}
    \item \textbf{Goal:} Find $T_{2s}$, $h_{2s}$, $x_{2s}$ (quality), $s_{2s}$
    \item \textbf{Process:}
    \begin{enumerate}
        \item Look up inlet state properties (superheated steam at $P_1 = 3$ MPa, $T_1 = 400^\circ$C): $h_1$, $s_1$ (from steam tables)
        \item Apply isentropic condition: $s_{2s} = s_1$
        \item At $P_2 = 100$ kPa, $s_{2s} = s_1$: compare $s_{2s}$ with $s_f$ and $s_g$ at 100 kPa
        \item If $s_f < s_{2s} < s_g$, two-phase mixture: $x_{2s} = (s_{2s} - s_f)/s_{fg}$
        \item Find enthalpy: $h_{2s} = h_f + x_{2s} h_{fg}$
        \item Temperature: $T_{2s} = T_{sat}$ at 100 kPa
    \end{enumerate}
    \item \textbf{Summary for Part A:}
    \begin{itemize}
        \item $h_1 = 3231.7$ kJ/kg, $s_1 = 6.923$ kJ/kg-K (inlet properties)
        \item $T_{2s} = 372.8$ K = $99.6^\circ$C
        \item $h_{2s} = 2512.6$ kJ/kg
        \item $x_{2s} = 0.928$ (92.8\% vapor)
        \item $s_{2s} = 6.923$ kJ/kg-K (isentropic)
    \end{itemize}
\end{itemize}

\textbf{PART B: Real Operation}

\begin{itemize}
    \item \textbf{Goal:} Find $T_2$, $h_2$, $x_2$, $s_2$ (actual outlet conditions)
    \item \textbf{Process:}
    \begin{enumerate}
        \item Use isentropic efficiency: $\eta_s = (h_1 - h_2)/(h_1 - h_{2s}) = 0.85$
        \item Solve for $h_2$: $h_2 = h_1 - \eta_s (h_1 - h_{2s})$
        \item At $P_2 = 100$ kPa, $h_2$ known: compare $h_2$ with $h_f$ and $h_g$ at 100 kPa
        \item If $h_f < h_2 < h_g$, two-phase mixture: $x_2 = (h_2 - h_f)/h_{fg}$
        \item Find entropy: $s_2 = s_f + x_2 s_{fg}$
        \item Temperature: $T_2 = T_{sat}$ at 100 kPa
    \end{enumerate}
    \item \textbf{Summary for Part B:}
    \begin{itemize}
        \item Isentropic work: $w_s = h_1 - h_{2s} = 3231.7 - 2512.6 = 719.1$ kJ/kg
        \item Actual work: $w_{actual} = \eta_s \times w_s = 0.85 \times 719.1 = 611.2$ kJ/kg
        \item Actual outlet enthalpy: $h_2 = h_1 - w_{actual} = 3231.7 - 611.2 = 2620.5$ kJ/kg
        \item $T_2 = 372.8$ K = $99.6^\circ$C
        \item $h_2 = 2620.5$ kJ/kg
        \item $x_2 = 0.976$ (97.6\% vapor)
        \item $s_2 = 7.213$ kJ/kg-K
    \end{itemize}
\end{itemize}

\textbf{PART C: Work Comparison}

To calculate and compare the turbine work output, we first find the isentropic work: $w_s = h_1 - h_{2s} = 3231.7 - 2512.6 = 719.1$ kJ/kg. The actual work is $w_{actual} = h_1 - h_2 = 3231.7 - 2620.5 = 611.2$ kJ/kg.

The corresponding power outputs are:
\begin{align*}
    \dot{W}_s &= \dot{m} w_s = 2.5 \times 719.1 = \boxed{1797.8 \text{ kW}} \\
    \dot{W}_{actual} &= \dot{m} w_{actual} = 2.5 \times 611.2 = \boxed{1528.0 \text{ kW}}
\end{align*}

The difference is $\Delta \dot{W} = \dot{W}_s - \dot{W}_{actual} = 1797.8 - 1528.0 = \boxed{269.8 \text{ kW}}$. This verifies the calculation since $w_{actual} = \eta_s \times w_s = 0.85 \times 719.1 = 611.2$ kJ/kg.

The real turbine produces 15\% less work due to irreversibilities (friction, turbulence, heat transfer within fluid), with the energy lost to entropy generation.

\textbf{PART D: Entropy Generation Rate}

\begin{itemize}
    \item \textbf{Goal:} Calculate $\dot{S}_{gen}$ using entropy balance
    \item \textbf{Given entropy balance:}
    \begin{align*}
        \sum \frac{\dot{Q}}{T_{boundary}} + \sum \dot{m} s_{in} - \sum \dot{m} s_{out} + \dot{S}_{gen} = \frac{dS_{sys}}{dt}
    \end{align*}
    \item \textbf{For steady flow, well-insulated turbine:}
    \begin{itemize}
        \item $\frac{dS_{sys}}{dt} = 0$ (steady state)
        \item $\sum \frac{\dot{Q}}{T_{boundary}} = 0$ (adiabatic)
        \item Single inlet and outlet stream
    \end{itemize}
    \item \textbf{Simplified:}
    \begin{align*}
        0 + \dot{m} s_1 - \dot{m} s_2 + \dot{S}_{gen} = 0 \\
        \Rightarrow \dot{S}_{gen} = \dot{m}(s_2 - s_1)
    \end{align*}
    \item \textbf{Calculation:}
    \begin{align*}
        \dot{S}_{gen} &= \dot{m}(s_2 - s_1) = 2.5 \times (7.213 - 6.923) \\
        &= 2.5 \times 0.290 = \boxed{0.725 \text{ kW/K}}
    \end{align*}
    \item \textbf{Check:} $\dot{S}_{gen} = 0.725 > 0$ (confirms 2nd law for real process)
\end{itemize}

\textbf{PART E: Comparison and Benefit}

Comparing the isentropic (A) versus real (B) turbine properties:

\begin{center}
\begin{tabular}{l|c|c|c}
Property & Isentropic (A) & Real (B) & Difference \\
\hline
$h_2$ (kJ/kg) & 2512.6 & 2620.5 & +107.9 \\
$s_2$ (kJ/kg-K) & 6.923 & 7.213 & +0.290 \\
$x_2$ (quality) & 0.928 & 0.976 & +0.048 \\
$T_2$ (K) & 372.8 & 372.8 & 0 \\
\end{tabular}
\end{center}

The key observation is that the real turbine has higher quality ($x_2 > x_{2s}$), higher entropy ($s_2 > s_1$), and higher enthalpy at exit ($h_2 > h_{2s}$). This provides a benefit: less liquid content in the exhaust (higher quality means more vapor), which reduces water droplet erosion on turbine blades. The non-ideal operation trades work output for blade lifetspan.

% \textbf{Solution Steps Summary:} (1) Look up inlet properties from steam tables, (2) Part A: Use $s_{2s} = s_1$ to find two-phase properties at $P_2$, (3) Part B: Apply $\eta_s$ to find $h_2$ and two-phase properties, (4) Part C: Calculate work from enthalpy drops and compare, (5) Part D: Apply entropy balance, (6) Part E: Compare moisture content and explain blade erosion benefit.


\newpage
\section*{Problem 11: Steam Quality Change in Piston-Cylinder}
\addcontentsline{toc}{section}{Entropy Change of a Pure Substance}
\vspace{0.5cm}

\textbf{Given:}
\begin{itemize}
    \item System: Free piston-cylinder (constant pressure process)
    \item Working fluid: $m = 3\ \mathrm{kg}$ of water/steam mixture
    \item Pressure: $P = 200\ \mathrm{kPa}$ (constant throughout process)
    \item Initial state: $x_1 = 0.60$ (40\% liquid, 60\% vapor)
    \item Final state: $x_2 = 1.0$ (all liquid evaporated to saturated vapor)
    \item Process: Adiabatic heating with electric resistance heater
\end{itemize}


\textbf{Relevant Equations}
\begin{itemize}
    \item For specific entropy in a two-phase mixture:
    \begin{align*}
        s &= s_f + x \cdot s_{fg}
    \end{align*}
    \item Specific entropy change:
    \begin{align*}
        \Delta s &= s_2 - s_1
    \end{align*}
\end{itemize}

For this entropy calculation, the initial state (1) has specific entropy $s_1 = s_f + x_1 \cdot s_{fg} = s_f + (0.60) \cdot s_{fg}$, while the final state (2) has $s_2 = s_g$ since it's saturated vapor. The change in specific entropy is therefore $\Delta s = s_2 - s_1 = s_g - [s_f + (0.60) \cdot s_{fg}]$.

\textbf{Solution}

From saturated steam properties at $P = 200$ kPa: $s_f = 1.530$ kJ/kg$\cdot$K, $s_g = 7.127$ kJ/kg$\cdot$K, and $s_{fg} = s_g - s_f = 5.597$ kJ/kg$\cdot$K.

The initial specific entropy is:
\begin{align*}
    s_1 &= s_f + x_1 \cdot s_{fg} = 1.530 + (0.60)(5.597) \\
    &= 1.530 + 3.358 = \boxed{4.888\ \mathrm{kJ/kg \cdot K}}
\end{align*}

The final specific entropy is $s_2 = s_g = 7.127\ \mathrm{kJ/kg \cdot K}$ for saturated vapor.

Therefore, the specific entropy change is:
\begin{align*}
    \Delta s &= s_2 - s_1 = 7.127 - 4.888 = \boxed{2.239\ \mathrm{kJ/kg \cdot K}}
\end{align*}

The positive $\Delta s$ reflects increased molecular disorder as liquid evaporates to vapor.


\section*{Problem 12: Piston-Cylinder Total Entropy Change}
\addcontentsline{toc}{section}{Total Entropy Change in a Piston-Cylinder}
\vspace{0.5cm}

\textbf{Given:}
\begin{itemize}
    \item System: Well-insulated piston-cylinder with freely moving piston
    \item Working fluid: Water (saturated liquid initially)
    \item Initial state: $V_1 = 4.8\ \mathrm{L}$, $P_1 = 150\ \mathrm{kPa}$, $x_1 = 0$
    \item Energy input: $Q_{in} = 1700\ \mathrm{kJ}$ from electric resistance heater
    \item Process: Constant pressure heating (piston moves freely)
\end{itemize}


\textbf{Assumptions}
\begin{itemize}
    \item Well-insulated system $\rightarrow$ no heat loss to surroundings
    \item Piston moves freely $\rightarrow$ constant pressure process
    \item Neglect kinetic and potential energy changes
    \item Quasi-equilibrium process
\end{itemize}

\textbf{Relevant Equations}
\begin{itemize}
    \item Energy Balance (First Law):
    \begin{align*}
        Q_{in} - W_{out} &= \Delta U = m(u_2 - u_1)
    \end{align*}
    For constant pressure:
    \begin{align*}
        W_{out} &= P(V_2 - V_1) = mP(v_2 - v_1)
        \\
        Q_{in} &= m(u_2 - u_1) + mP(v_2 - v_1) = m(h_2 - h_1)
    \end{align*}
    \item Entropy Change:
    \begin{align*}
        \Delta S &= m(s_2 - s_1) \\
        \Delta s &= s_2 - s_1
    \end{align*}
\end{itemize}

\textbf{Solution}

The initial state properties at 150 kPa saturated liquid are: $h_1 = h_f = 467.1$ kJ/kg, $s_1 = s_f = 1.434$ kJ/kg$\cdot$K, $\rho_1 = 949.9$ kg/m$^3$, and specific volume $v_1 = 1/\rho_1 = 0.001053$ m$^3$/kg.

The mass of water is:
\begin{align*}
    m = \frac{V_1}{v_1} = \frac{0.0048\ \text{m}^3}{0.001053\ \text{m}^3/\text{kg}} = \boxed{4.558\ \text{kg}}
\end{align*}

Applying the energy balance for constant pressure:
\begin{align*}
    Q_{in} &= m(h_2 - h_1) \\
    h_2 &= h_1 + \frac{Q_{in}}{m} = 467.1 + \frac{1700}{4.558} \\
    &= 467.1 + 373.0 = \boxed{840.1\ \text{kJ/kg}}
\end{align*}

Since $h_1 = 467.1 < h_2 = 840.1 < h_g = 2693.1$ kJ/kg (saturated vapor enthalpy), the final state is a two-phase mixture with quality $x_2 = 0.168$ (16.8\% vapor) and final entropy $s_2 = 2.404$ kJ/kg$\cdot$K.

The entropy changes are:
\begin{align*}
    \Delta s &= s_2 - s_1 = 2.404 - 1.434 = \boxed{0.970\ \text{kJ/kg$\cdot$K}} \\
    \Delta S &= m \times \Delta s = 4.558 \times 0.970 = \boxed{4.42\ \text{kJ/K}}
\end{align*}




\section*{Problem 13: Solar Energy Storage Entropy Analysis}
\addcontentsline{toc}{section}{Entropy Changes: Incompressible Substances and Ideal Gases}
\vspace{0.5cm}

\textbf{Given:}
\begin{itemize}
    \item \textbf{Ceramic Particles (CARBOBEAD CP):}
    \begin{itemize}
        \item Density: $\rho = 3270\ \mathrm{kg/m^3}$
        \item Specific heat: $C = 1.3\ \mathrm{kJ/kg\cdot K}$
        \item Temperature range: $T_1 = 250^\circ\mathrm{C}$ to $T_2 = 1100^\circ\mathrm{C}$
    \end{itemize}
    \item \textbf{Compressed Air System:}
    \begin{itemize}
        \item Mass flow rate: $\dot{m} = 0.75\ \mathrm{kg/s}$
        \item Inlet conditions: $T_3 = 250^\circ\mathrm{C}$, $P_3 = 4000\ \mathrm{kPa}$
        \item Heating: $T_4 = 1100^\circ\mathrm{C}$ at constant pressure
        \item Turbine expansion: Isentropic to $P_5 = 100\ \mathrm{kPa}$
    \end{itemize}
    \item \textbf{Air Properties:} Gas constant $R = 0.287\ \mathrm{kJ/kg\cdot K}$
\end{itemize}

\textbf{Assumptions:}
\begin{itemize}
    \item \textbf{Ceramic Particles (CARBOBEAD CP):}
    \begin{itemize}
        \item Incompressible solid with constant specific heat ($C = 1.3$ kJ/kg$\cdot$K)
        \item Uniform heating from $250^\circ\mathrm{C}$ to $1100^\circ\mathrm{C}$
        \item Negligible thermal expansion effects
    \end{itemize}
    \item \textbf{Compressed Air System:}
    \begin{itemize}
        \item Ideal gas behavior with constant gas constant ($R = 0.287$ kJ/kg$\cdot$K)
        \item Negligible pressure drop in solar receiver ($P_4 = P_3 = 4000$ kPa)
        \item Constant pressure heating process (3$\rightarrow$4)
        \item Reversible adiabatic (isentropic) turbine expansion (4$\rightarrow$5)
        \item Steady flow conditions throughout
    \end{itemize}
    \item \textbf{Reference State:} Table A-17 reference at 0 K and 100 kPa
    \item \textbf{Negligible Effects:} Kinetic and potential energy changes, heat losses
\end{itemize}

\textbf{PART A: Change in specific entropy (kJ/kg$\cdot$K) of ceramic particles from 1 $\rightarrow$ 2}

For an incompressible substance with constant specific heat:
\begin{align*}
    \Delta s_{1\rightarrow2} &= C \ln\left(\frac{T_2}{T_1}\right) \\
    &= 1.3\ \mathrm{kJ/kg \cdot K} \times \ln\left(\frac{1373.15}{523.15}\right) \\
    &= 1.3 \times \ln(2.625) = 1.3 \times 0.965 \\
    &= \boxed{1.254\ \mathrm{kJ/kg \cdot K}}
\end{align*}

\textbf{PART B: Change in specific entropy (kJ/kg$\cdot$K) of compressed air from 3 $\rightarrow$ 4}

Using CoolProp for air property calculations with state 3 at $T_3 = 523.15$ K, $P_3 = 4000$ kPa and state 4 at $T_4 = 1373.15$ K, $P_4 = 4000$ kPa (constant pressure heating):

\begin{align*}
    \Delta s_{3\rightarrow4} &= s_4 - s_3 \\
    s_3 &= 3389.9\ \mathrm{J/kg \cdot K} = 3.390\ \mathrm{kJ/kg \cdot K} \\
    s_4 &= 4468.7\ \mathrm{J/kg \cdot K} = 4.469\ \mathrm{kJ/kg \cdot K} \\
    \Delta s_{3\rightarrow4} &= 4.469 - 3.390 = \boxed{1.079\ \mathrm{kJ/kg \cdot K}}
\end{align*}

\textbf{PART C: Specific entropy (kJ/kg$\cdot$K) of air at state 4}

From Part B calculation:
\begin{align*}
    s_4 = \boxed{4.469\ \mathrm{kJ/kg \cdot K}}
\end{align*}

\textbf{PART D: Work (kW) produced via turbine from 4 $\rightarrow$ 5}

For isentropic expansion ($s_5 = s_4$), we solve for $T_5$ at $P_5 = 100$ kPa using property\_solver.py to find $T_5 = 529.7$ K where $s_5 = 4468.7$ J/kg$\cdot$K.

Using CoolProp for enthalpy property lookup:
\begin{align*}
    h_4 &= 1612.8\ \mathrm{kJ/kg} \quad \text{(at 1373.15 K, 4000 kPa)} \\
    h_5 &= 660.1\ \mathrm{kJ/kg} \quad \text{(at 529.7 K, 100 kPa)} \\
    w_t &= h_4 - h_5 = 1612.8 - 660.1 = 952.7\ \mathrm{kJ/kg} \\
    \dot{W}_{turb} &= \dot{m} \times w_t = 0.75 \times 952.7 = \boxed{714.5\ \mathrm{kW}}
\end{align*}

% \textbf{Solution Summary}
% \begin{enumerate}
%     \item \textbf{Ceramic Particles:} $\Delta s_{1\rightarrow2} = 1.254$ kJ/kg$\cdot$K (heating from 250$^\circ$C to 1100$^\circ$C)
%     \item \textbf{Air Entropy Change:} $\Delta s_{3\rightarrow4} = 1.079$ kJ/kg$\cdot$K (constant pressure heating)
%     \item \textbf{Air Entropy at State 4:} $s_4 = 4.469$ kJ/kg$\cdot$K 
%     \item \textbf{Turbine Work:} $\dot{W}_{turb} = 714.5$ kW (isentropic expansion from 4000 kPa to 100 kPa)
% \end{enumerate}

% \textbf{Key Tools Used:}
% \begin{itemize}
%     \item \textbf{CoolProp} For accurate air property calculations instead of Table A-17 approximations
%     \item \textbf{Property Solver:} Iterative tool to find $T_5$ where $s_5 = s_4$ for isentropic expansion
%     \item \textbf{Incompressible Substance Formula:} $\Delta s = C \ln(T_2/T_1)$ for ceramic particles
% \end{itemize}



\newpage

\section*{Problem 14: Concentrated Solar Power Carnot Analysis}
\addcontentsline{toc}{section}{Carnot Cycle Heat Engines}
\vspace{0.5cm}

\textbf{Given:}
\begin{itemize}
    \item \textbf{Solar Concentrator System:}
    \begin{itemize}
        \item Concentration ratios: $C = 100, 500, 1000, 2000, 3000$ suns
        \item Solar irradiance: $I = 1000\ \mathrm{W/m^2}$
        \item Stefan-Boltzmann constant: $\sigma = 5.67 \times 10^{-8}\ \mathrm{W/(m^2\cdot K^4)}$
    \end{itemize}
    \item \textbf{Carnot Heat Engine:}
    \begin{itemize}
        \item Cold reservoir: $T_L = 300\ \mathrm{K}$ (ambient temperature)
        \item Hot reservoir: $T_H$ varies (300 K to 2600 K)
    \end{itemize}
\end{itemize}

\textbf{Assumptions:}
\begin{itemize}
    \item \textbf{Ideal Carnot cycle:} Reversible processes, no irreversibilities in heat engine
    \item \textbf{Steady-state operation:} Constant solar irradiance and ambient conditions
    \item \textbf{Perfect solar concentration:} All concentrated solar energy reaches receiver surface
    \item \textbf{Black-body radiation:} Receiver emits according to Stefan-Boltzmann law
    \item \textbf{Ambient cold reservoir:} $T_L = 300$ K (environmental heat sink)
    \item \textbf{Negligible convection losses:} Only radiative losses from receiver considered
    \item \textbf{Uniform receiver temperature:} Isothermal hot reservoir at $T_H$
\end{itemize}

\textbf{PART A: Plot Total Efficiency}

\begin{figure}[H]
    \centering
    \includegraphics[width=0.9\textwidth]{../images/carnot_efficiency_plot.pdf}
    \caption{Total system efficiency vs hot reservoir temperature for different concentration ratios. Optimal operating points are marked with circles.}
    \label{fig:carnot_efficiency}
\end{figure}

\begin{figure}[H]
    \centering
    \includegraphics[width=0.9\textwidth]{../images/carnot_components_plot.pdf}
    \caption{Component efficiencies: (top) Solar receiver efficiency decreases with temperature due to radiation losses, (bottom) Carnot heat engine efficiency increases with temperature.}
    \label{fig:carnot_components}
\end{figure}

\textbf{Relevant Equations}
\begin{itemize}
    \item Receiver Efficiency:
    \begin{align*}
        \eta_{receiver} &= 1 - \frac{\sigma T_H^4}{C \cdot I}
    \end{align*}
    \item Carnot Heat Engine Efficiency:
    \begin{align*}
        \eta_{Carnot,HE} &= 1 - \frac{T_L}{T_H}
    \end{align*}
    \item Total System Efficiency:
    \begin{align*}
        \eta_{Total} &= \eta_{receiver} \cdot \eta_{Carnot,HE} \\
        &= \left(1 - \frac{5.67 \times 10^{-8} \cdot T_H^4}{C \cdot 1000}\right) \cdot \left(1 - \frac{300}{T_H}\right)
    \end{align*}
\end{itemize}

\textbf{Solution using Python Analysis}

\begin{itemize}
    \item \textbf{Python Scripts:} carnot\_simple.py for optimization, carnot\_plotter.py for visualization
    \item \textbf{Method:} Temperature optimization scan to find maximum $\eta_{total}$ for each concentration ratio
\end{itemize}

\textbf{Results - Optimal Operating Conditions:}

\begin{center}
\begin{tabular}{c|c|c|c|c|c}
\toprule
$C$ [suns] & $T_{H,opt}$ [K] & $T_{H,opt}$ [$^\circ$C] & $\eta_{total,max}$ & $\eta_{receiver}$ & $\eta_{Carnot}$ \\
\midrule
100 & 719 & 446 & 49.4\% & 84.8\% & 58.3\% \\
500 & 971 & 698 & 62.1\% & 89.9\% & 69.1\% \\
1000 & 1107 & 834 & 66.7\% & 91.5\% & 72.9\% \\
2000 & 1263 & 990 & 70.7\% & 92.8\% & 76.2\% \\
3000 & 1366 & 1093 & 72.9\% & 93.4\% & 78.0\% \\
\bottomrule
\end{tabular}
\end{center}

\textbf{Sample Data for Plotting:}

\begin{center}
\begin{tabular}{c|c|c|c|c|c}
\toprule
$T_H$ [K] & $C=100$ & $C=500$ & $C=1000$ & $C=2000$ & $C=3000$ \\
\midrule
500 & 0.386 & 0.397 & 0.399 & 0.399 & 0.400 \\
750 & 0.492 & 0.578 & 0.589 & 0.595 & 0.596 \\
1000 & 0.303 & 0.621 & 0.660 & 0.680 & 0.687 \\
1250 & -- & 0.550 & 0.655 & 0.707 & 0.725 \\
1500 & -- & 0.341 & 0.570 & 0.685 & 0.723 \\
1750 & -- & -- & 0.388 & 0.608 & 0.682 \\
2000 & -- & -- & 0.079 & 0.464 & 0.593 \\
\bottomrule
\end{tabular}
\end{center}

\textbf{PART B: Identify Optimal Receiver Temperature}

\textbf{Key Trends Observed:}
\begin{enumerate}
    \item \textbf{Optimal Temperature Increases with Concentration:}
    \begin{itemize}
        \item $C = 100$ suns: $T_{H,opt} = 719$ K (446$^\circ$C)
        \item $C = 3000$ suns: $T_{H,opt} = 1366$ K (1093$^\circ$C)
        \item Higher concentration allows operation at higher temperatures
    \end{itemize}
    
    \item \textbf{Maximum Efficiency Increases with Concentration:}
    \begin{itemize}
        \item $C = 100$ suns: $\eta_{max} = 49.4\%$
        \item $C = 3000$ suns: $\eta_{max} = 72.9\%$
        \item Improvement of 23.5 percentage points
    \end{itemize}
    
    \item \textbf{Physical Trade-offs:} Carnot efficiency increases with hot reservoir temperature, while receiver efficiency decreases due to radiative losses scaling as $T_H^4$. The optimal operating point balances these competing effects, and higher concentration ratios shift this optimum to higher temperatures and system efficiencies.
\end{enumerate}

\newpage
% \textbf{Python Code Implementation}

% The analysis was performed using custom Python scripts for accurate calculations:

% \begin{verbatim}
% #!/usr/bin/env python3
% """
% Problem 14: Carnot Cycle Heat Engines - Analysis
% """

% import math
% import numpy as np
% import matplotlib.pyplot as plt

% # Physical constants
% SIGMA = 5.67e-8  # Stefan-Boltzmann constant [W/(m^2\cdot K^4)]
% I = 1000  # Solar irradiance [W/m^2]
% T_L = 300  # Cold reservoir temperature [K]

% # Concentration ratios [suns]
% concentration_ratios = [100, 500, 1000, 2000, 3000]

% def receiver_efficiency(T_H, C):
%     """Calculate receiver efficiency: eta_receiver = 1 - sigma*T_H^4/(C*I)"""
%     return 1 - (SIGMA * T_H**4) / (C * I)

% def carnot_efficiency(T_H, T_L):
%     """Calculate Carnot heat engine efficiency: eta_Carnot = 1 - T_L/T_H"""
%     return 1 - T_L / T_H

% def total_efficiency(T_H, C, T_L):
%     """Calculate total system efficiency: eta_total = eta_receiver x eta_Carnot"""
%     eta_recv = receiver_efficiency(T_H, C)
%     eta_carnot = carnot_efficiency(T_H, T_L)
%     if eta_recv <= 0:  # Invalid operating regime
%         return 0
%     return eta_recv * eta_carnot

% def find_optimal_temperature(C, T_L):
%     """Find optimal temperature by scanning the valid range"""
%     T_range = np.arange(T_L + 50, 2601, 1)  # 350K to 2600K
    
%     max_efficiency = -1
%     optimal_T_H = T_L + 50
    
%     for T_H in T_range:
%         eta = total_efficiency(T_H, C, T_L)
%         if eta > max_efficiency:
%             max_efficiency = eta
%             optimal_T_H = T_H
    
%     return float(optimal_T_H), max_efficiency

% # Find optimal conditions for each concentration ratio
% for C in concentration_ratios:
%     T_opt, eta_opt = find_optimal_temperature(C, T_L)
%     print(f"C = {C:4d} suns: T_opt = {T_opt:6.1f} K, "
%           f"eta_max = {eta_opt:5.1%}")
% \end{verbatim}

% \textbf{Command Line Execution:}
% \begin{verbatim}
% $ python carnot_simple.py      # Analysis and optimization
% $ python carnot_plotter.py     # Generate efficiency plots
% \end{verbatim}

% \textbf{Key Insights from Analysis:}

% \begin{enumerate}
%     \item \textbf{Physical Trade-offs:}
%     \begin{itemize}
%         \item \textbf{Carnot efficiency} increases with $T_H$ (better thermodynamic limit)
%         \item \textbf{Receiver efficiency} decreases with $T_H$ (radiation losses $\propto T_H^4$)
%         \item Optimum occurs where product is maximized
%     \end{itemize}
    
%     \item \textbf{Concentration Effects:}
%     \begin{itemize}
%         \item Higher $C$ shifts optimum to higher temperatures
%         \item Concentrated sunlight ($C \cdot I$) can overcome larger radiation losses ($\sigma T_H^4$)
%         \item Engineering trade-off: system complexity vs. performance
%     \end{itemize}
% \end{enumerate}

% \textbf{Physical Interpretation}

% The total efficiency is the product of two competing effects:
% \begin{align*}
%     \eta_{total}(T_H) = \underbrace{\left(1 - \frac{\sigma T_H^4}{C \cdot I}\right)}_{\text{decreases with } T_H} \times \underbrace{\left(1 - \frac{T_L}{T_H}\right)}_{\text{increases with } T_H}
% \end{align*}

% At the optimum: $\frac{d\eta_{total}}{dT_H} = 0$, which yields the balance between thermodynamic benefits and radiation losses.

% \textbf{Python Command Used:}
% \begin{itemize}
%     \item \texttt{python carnot\_simple.py} $\rightarrow$ Complete analysis with optimization
% \end{itemize}




% ========== APPENDIX ========== 
% \appendix
% \section{Appendix}
% Add any supplementary material, extra figures, or code listings here.

% ========== IMAGE, TABLE, AND CODE EXAMPLES ========== 

% Example image:
% \begin{figure}[H]
%     \centering
%     \includegraphics[width=0.6\textwidth]{your-image-file}
%     \caption{Caption for your image.}
%     \label{fig:example}
% \end{figure}

% Example table:
% \begin{table}[H]
%     \centering
%     \begin{tabularx}{0.8\textwidth}{l X}
%         \toprule
%         Header 1 & Header 2 \\
%         \midrule
%         Row 1 & Description \\
%         Row 2 & Description \\
%         \bottomrule
%     \end{tabularx}
%     \caption{Example table.}
%     \label{tab:example}
% \end{table}

% Example code snippet:
% \begin{verbatim}
% % Paste your code here
% for i = 1:10
%     disp(i)
% end
% \end{verbatim}

\end{document}


