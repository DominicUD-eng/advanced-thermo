\documentclass[12pt]{article}

% ========== PACKAGES ========== 
% Math and equations
\usepackage{amsmath}
\usepackage{amssymb}

% Graphics and images
\usepackage{graphicx}
\usepackage{float}

% Tables
\usepackage{booktabs}
\usepackage{tabularx}
\usepackage{array}

% Layout and formatting
\usepackage{geometry}
\usepackage{setspace}
\usepackage{fancyhdr}
\usepackage{pdflscape}  % For landscape orientation
\usepackage{xcolor}     % For colors
\usepackage{colortbl}   % For colored table rows

% Other utilities
\usepackage{hyperref}

% ========== DOCUMENT SETUP ========== 
% Margins (1 inch on all sides)
\geometry{margin=1in}

% Single spacing
\singlespacing

% Header and footer
\pagestyle{fancy}
\fancyhf{}
\rhead{\thepage}

% ========== COVER PAGE ========== 
\begin{document}

\begin{titlepage}
    \centering
    \vspace*{2cm}
    {\LARGE\bfseries Advanced Thermodynamics Homework\\[1.5cm]}
    {\large Assignment: \underline{\hspace{6cm}} \\[1cm]}
    {\large Name: \underline{\hspace{7cm}} \\[1cm]}
    {\large Date: \underline{\hspace{7cm}} \\[2cm]}
\end{titlepage}

% Table of contents
\tableofcontents
\newpage


% ========== PROBLEM-BY-PROBLEM HOMEWORK TEMPLATE ========== 

% Instructions: For each homework problem, copy the block below and increment the problem number.
% Each problem will start on a new page.

\section*{Problem 1}
\addcontentsline{toc}{section}{Problem 1}
\vspace{0.5cm}
% Write your solution to Problem 1 here.

\newpage

\section*{Problem 2}
\addcontentsline{toc}{section}{Problem 2}
\vspace{0.5cm}
% Write your solution to Problem 2 here.

\newpage

% Add more problems as needed by copying the block above and updating the problem number.


% ========== APPENDIX ========== 
\appendix
\section{Appendix}
% Add any supplementary material, extra figures, or code listings here.

% ========== IMAGE, TABLE, AND CODE EXAMPLES ========== 

% Example image:
% \begin{figure}[H]
%     \centering
%     \includegraphics[width=0.6\textwidth]{your-image-file}
%     \caption{Caption for your image.}
%     \label{fig:example}
% \end{figure}

% Example table:
% \begin{table}[H]
%     \centering
%     \begin{tabularx}{0.8\textwidth}{l X}
%         \toprule
%         Header 1 & Header 2 \\
%         \midrule
%         Row 1 & Description \\
%         Row 2 & Description \\
%         \bottomrule
%     \end{tabularx}
%     \caption{Example table.}
%     \label{tab:example}
% \end{table}

% Example code snippet:
% \begin{verbatim}
% % Paste your code here
% for i = 1:10
%     disp(i)
% end
% \end{verbatim}

\end{document}
